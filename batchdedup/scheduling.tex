\section{Round Scheduling}
\label{sect:scheduling}
The simplest way to take advantage of the efficiency of batch processing is to
schedule all the work to be done in one round.  This works well if all of the
data being backed up is just copies (e.g. a separate backup system which data
is sent to over the network), however in our case we are backing up the
original virtual disk, which may still be in use during the snapshot. This adds
extra complexity to our cost analysis, because we must now also consider the
cost of maintaining a constistent view during the snapshot process. We use the
Copy on Write (CoW) provided by the virtual disk manager. With CoW, the
duration of the backup affects how much data must be copied. Other studies have
shown that as much as 8\% or even more of total capacity must be reserved for
Cow \cite{EMCIncrementalDataChanges}. The actual cost of CoW is a factor of the
data size, the write rate, and duration CoW is taking place. The data size
isn't something we can change, nor the write rate, but we can minimize the
duration that a given VM is undergoing CoW. We assume a poisson distribution of
writes (which closely fits the meaured results from
\cite{EMCIncrementalDataChanges}), and then try to minimize the CoW cost using
our performance and CoW model. Although the single batch schedule completes the
backup in the smallest amount of time, it also has the greatest CoW cost
because the most processing must be done before any VM can release the CoW
lock.

The way to decrease the CoW cost is to break up the backup into multiple
rounds, where in each round the CoW cost is minimized. The more rounds there
are the shorter each round can be and therefore the smaller the CoW cost. The
more rounds there are however the greater the backup overheads and some of the
efficiency gained from batch processing is lessened. We balance these costs by
setting a time limit on the whole backup job, and then develop an algorithm to
schedule VMs into rounds.

With these goals a model that closely fits our goals is the dual version of the
bin packing problem. In standard bin packing the goal is to fit all of the
items into as few bins as possible, without overfilling any bins. In the dual
version of the problem however as many bins as possible are to filled to at
least some minimum level. In our problem the constraint is to keep the total
cost of the schedule under a time limit rather than a minimum bin level. We
adapt an algorithm for dual bin packing\cite{DualBinPacking} to fit our VM
scheduling problem. The algorithm adapted is called iterated A and works by
iteratively callng a bin packing heuristic A with the VMs to be scheduled and
the number of rounds, using binary search to arrive at the best number of
rounds. A(I,N) is defined to return the optimality of packing set I into N bins
using A. We take this basic idea and look at several VM packing heuristics to
arrive at an efficient packing algorithm. More formally, our adaptation of the
iterated A algorithm can be defined as:

\begin{lstlisting}
Set UB=min(n,2*v)
Set LB=1
while UB>LB
    set N = (UB+LB+1)/2
    if A(machines,N) > T, set UB=N-1
    else set LB=N
Halt
\end{lstlisting}
where A(I,N) returns the total backup time of the schedule\\
return packing generated by A(machines,UB) after loop finishes

This general algorithm relies on a good choice of A to arrive at an efficent
packing. Our first VM packing heuristic, A0, is very close to the algorithm for
the dual bin packing paper.

A0:
\begin{lstlisting}
sort VMs in descending order by size
while there is an unscheduled VM
    pick the first unscheduled VM a
    pick the round with the current
        lowest runtime b
        tie-breaker:left-most round
    schedule VM a to round b
Halt
\end{lstlisting}

A0 decreases CoW cost (see Table~\ref{tab:schedule-costs}), but has room for
improvement. The first issue is that the round with the current lowest runtime
is chosen.  Because backup time is dependent on a combination of the highest
machine load and average machine load, if we add a VM to the currently most
loaded machine in a round it will increase runtime much more than if we add the
VM to a currently empty machine on an equally heavily loaded round
\textit{(maybe a figure to show this?)}. Therefore a better scheduling would be
obtained if we pick the round with the lowest runtime after we simulate adding
the new VM to that round. This new round picking heuristic takes into account
that the same VM might have different affects on different rounds.

Another improvement we can make is to the VM picking heuristic.
