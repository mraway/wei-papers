\section{CDS Analysis}
Although locality based data reduction can remove most of the inner-VM duplications,
it's not able to solve the data duplication cross VMs. Different VMs still share large amount
of common data such that their snapshot backups would have a lot of data in common.
Our observations on Aliyun's real VM user data reveals several major sources of cross-VM data duplication:
  \begin{enumerate}
  \item \emph{System-related data}: These data are generated by public third-parties, they are copied/downloaded/installed into VM disks either automatically or manually. Once installed, operations on such data are mostly read only until software updates arrive. For example, data of operating systems, some widely-used software such as Apache and MySQL, and their documations all fall into this category.
  \item \emph{User-generated data}: These data are generated by individual user's behavior, either directly or indirectly. They are much less duplicated than the system-related data, but the zipf-like distribution indicates that a small amount of data in this category could represent most of data duplications.
  \item \emph{Zero-filled data}: Like previous studies [refs here], we've observed that zero-filled data exist pervasively at system wide. They are almost like the spaces and commas in text articles. Under content-based chunking, they were distilled as zero-filled blocks with the maximum allowed length. 
%Zero-filled blocks are mostly generated by user behavior to preserve the space for future data, thus we treat them as a special case of user-generated data.
  \end{enumerate}

Without eliminating the data duplication between VM snapshot backups, storage space is severely wasted when the number of VMs increases.
To address this problem, we developed a technique called \emph{Common Data Set (CDS)} to eliminate data duplication for all three situations above. 
CDS is a public data library that shared by all VM snapshot backups in the cluster, 
which consists of the most popular data blocks in VM snapshot backups. It is generated, 
managed and accessed in an uniform manner by all VMs such that [write some fancy system characteristic stuff here].

\subsection{CDS Size V.S. Coverage}
As a shared data library, we expect CDS to collect almost all the system related data, the most popular user-generated data and the zero-filled data block.
With CDS as a public data reference, each VM's snapshot backup has no need to store its own copies for data that can be found in CDS, instead they can just
store a reference.

The more data we put into CDS, the closer we come to complete deduplication. But in reality we are facing a list of restrictions such like [blabla].
We borrowed the idea of web caching[refs here] to analysis the size of CDS vesus its effectiveness on data reduction.

We let $M$ be the number of nodes in a cluster, $N$ be the number of VMs hosted per node,
and let $D$ denotes the average size of one VM's snapshot backups (after level 1 and 2 reduction, which is near 40 GB in our production environment).
And let $C_0$, $C_s$ and $C_u$ denote the size of zero-filled block, system-related data and user-generated data in CDS,
then the total size of CDS is represented by $C=C_0+C_s+C_u$.
We let $S_0$, $S_s$ and $S_u$ denote the corresponding data coverage ratio (with respect to $D$) by $C_0$, $C_s$ and $C_u$,
so the total space saving ratio is $S=S_0+S_s+S_u$.

Zero-filled block at maximum length is the first item we need to put into CDS. This is the one and only special data block, 
and because CDS only stores unique data blocks, $C_0$ is almost equal to 0. In practice we found zero-filled blocks account
for near 20\% of total data, so $S_0$ is 20\%.

The rarely modified system-related data are the second to be added to CDS. We have thousands of VMs in each cluster, 
but all these VMs fall into only a few OS types, using a limited selection of software, therefore data duplication in this category
is highly noticable in a global block counting. In particular, most of such data already exist in the VM base images. 
We analyzed VMs running various services from 7 mainstream OS types in our cluster, and found close to 50GB of such data in total. 
Because system-related data don't change frequently, it's safe to expect that for 20 OS variations 
and software updates in 2 years, $C_s$ will not grow to exceed 200 GB.
For each VM, from 2 to 20 GB of data can be reduced in this way, depends on the OS and service type, so on average we estimate
$S_s$ would be no less than 20\%.

The rest of data are user-generated, the total size of such data can be written as $D_u=D-S_0-S_s$, which is 60\% of $D$. 
It follows a zipf-like distribution with $\alpha$ between 0.65 to 0.7. 
Let $T_u$ be the total size of unique data blocks in $D_u$,
in practice we notice that complete deduplication for such data always result in a 2:1 reduction ratio,
, so $T_u=D_u/2$.
Since we collect the most popular user-generated data into CDS, the coverage of CDS on user-generated data
is $(C_u/T_u)^{1-\alpha}$.

The following table lists CDS coverage on user-generated data under different $\alpha$ and $C_u/T_u$.

[a table of coverage with alpha from 0.65-0.70, ratio from 0.002 0.005 0.01 0.02 0.05]

It's obviously that at least 30\% of user-generated data can be reduced in this way, using about 0.02 of $T_u$, or 0.01 of $D_u$.
The size of user-generated data in CDS would be 0.006$D$, with coverage $S_u=0.18D$.

Thus the estimation of CDS coverage is $S=S_0+S_s+S_u=0.58$, with the size of CDS to be no more than $(200 + 0.006*D*N)$ GB.
In a VM cluster of 100 machines, with 25 VMs on each physical machine, the size of CDS is 800 GB in total, or 8 GB per machine.
The total size of CDS index would be 8 GB, which will cost 80 MB memory on each machine. 

\subsection{CDS Access}

\subsection{CDS Management}

