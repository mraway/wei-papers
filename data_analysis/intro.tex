\section{Introduction}
In a virtualized cloud environment such as ones provided by Amazon EC2 and Aliyun,
each instance of a guest operating system runs on a virtual machine, accessing
virtual hard disks represented as virtual disk image files in the host operating system.
Because these image files are stored as regular files from the external point of view,
backing up VM's data is mainly done by taking snapshots of virtual disk images.

Frequent  backup of VM snapshots increases  the reliability of VM's hosted in a cloud.
For example, Aliyun, the largest cloud service provider by Alibaba in China, 
provides automatic frequent backup of VM images to strengthen the reliability of its service for all users.
The cost of frequent backup of VM snapshots is  high because of the huge storage demand.

Unlike the legacy backup systems\cite{jumbo07} dealing with general file-level backup and deduplication, 
Backing up VM images is different. Although each VM image is treated as a file logically, its size is very large.
On the other hand, a cloud must support parallel backup of a large number of virtual disks everyday. 
One key requirement we face at Aliyun is that VM snapshot backup should only use a minimal amount of system
resources so that most of resources is kept for regular cloud system services or applications.
Thus our objective is to exploit the characteristics and duplication pattern of VM snapshot data and
pursue a cost-effective deduplication solution. 

There are several previous studies on this topic. Jayaram\cite{Jayaram2011} and
Jin\cite{Jin2009} has investigated the data similarity between VM images using 
Rabin's fingerprinting\cite{identify00} algorithm. 
Silo\cite{xia2011} and Extreme Binning\cite{extreme_binning09} studied the problem
of deduplication in a large distributed environment which also helps solving the
problem of VM snapshot backup.

In this paper we present the data analysis of Aliyun's VM snapshot data.
Our work is different from previous studies at several aspects:
First, we are targeting at the problem of snapshot backups, 
and no previous study has studied VM image with backup data involved.
Second, we focus on observing the pattern of data duplication in VM snapshot backups,
rather than examining the effect of variable-sized chunking algorithm.
Finally, we use real user's VM data rather than hand-made VM images.

\section{Experiment Setup}
We sampled two data sets from Aliyun's public VM cluster, all VMs
are used by real external users running various applications such like
database, web server, rendering services or even Hadoop. Each VM has
two virtual disks, one is for OS and software installations, and the other one
is for storing user data contents.

Data set VOSS composes of the OS disks from 35 VMs in 7 popular OSes: 
Debian, Ubuntu, Redhat, CentOS, Win2003 32bit, win2003 64 bit and win2008 64 bit. For each OS, 
5 VMs are chosen, and every VM come with 10 full snapshots of it OS disk. So
there is 350 full snapshot backups in this data set, overall size is about 7 TB.
We use VOSS to study the backup duplication characteristics, and OS disk change patterns.

Data set DDS contains the first snapshots of 1323 VMs' data disks from a cluster with 100+ nodes. 
Since no backup duplication is involved in this data set, this data set helps us to 
study the duplication pattern of user generated data. The overall size of dataset2 is near 23 TB.