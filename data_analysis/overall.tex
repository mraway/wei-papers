\section{Overall Deduplication Effect}
Our study start from examining the effect of complete deduplication over both data sets.
Each virtual disk image is divided into small variable-sized blocks using the TTTD algorithm\cite{frame05},
under the average 4KB, maximum 16KB and minimum 4KB setting. Complete deduplication is
done by caculating the SHA-1 hash of each block and identifying duplicate copies base on compare-by-hash.

Regarding to the VM snapshots storage, one popular technique is to split disk image file into fix-sized
pages, and only store the dirty pages since last backup. But unlike complete deduplication considering duplicates
across VMs, this technique is limited within each VM's backups. We also tested this data reduction method
using 2MB page size over the dataset VOSS. 

Table~\ref{tab:dedup} shows the data reduction of both methods, 
the reduction ratio is defined as the original size divide by size after reduction.
In general, complete deduplication achieves great data reduction

We see that 

\begin{table*}[htb]
  \centering
    \begin{tabular}{|l|l|p{0.8in}|p{1.1in}|p{0.3in}|p{0.9in}|p{0.3in}|}
        \hline
        Data Set & OS Type & Original Size \newline (GB) & After Complete \newline Deduplication (GB) & Ratio & After 2MB Page \newline Reduction (GB) & Ratio \\ \hline
        \multirow{8}{*}{VOSS} & Debian & 1034.59 & 77.10 & 13.42 & 218.29 & 4.74 \\ \cline{2-7}
         & Ubuntu & 989.32 & 81.60 & 12.12 & 178.38 & 5.55 \\ \cline{2-7}
         & RHEL & 1007.28 & 62.70 & 16.07 & 215.33 & 4.68 \\ \cline{2-7}
         & CentOS & 973.03 & 57.34 & 16.97 & 522.53 & 1.86 \\ \cline{2-7}
         & Win2003 32Bit & 630.37 & 30.12 & 20.93 & 150.31 & 4.19 \\ \cline{2-7}
         & Win2003 64Bit & 793.47 & 44.16 & 17.97 & 167.15 & 4.75 \\ \cline{2-7}
         & Win2008 64Bit & 1508.97 & 46.39 & 32.52 & 222.54 & 6.78 \\ \cline{2-7}
         & Combined & 6937.03 & 389.65 & 17.80 & 1674.53 & 4.14 \\ \hline
        DDS & ~ & 23125.2 & 10887.2 & 2.12 & ~ & ~ \\
        \hline
    \end{tabular}
    \caption{Data reduction via complete deduplication and dirty page reduction}
    \label{tab:dedup}
\end{table*}

\emph{Observation 1: Both methods reduce the data significantly, 
but there is still a big gap between the effect of dirty page method 
and what complete deduplication can acheive.} 
We believe complete deduplication can reduces 2x~5x more data than dirty page method for several resons:
First, the dirty page method doesn't resolve the duplication across VM backups. Second, the page size
is usually much bigger than actual range of modification, so many unchanged data are still backed up.

\emph{Observation 2: Locality could be ruined with system upgrade.}
We notice dirty page method works poorly on CentOS, while complete deduplication works as efficient as on
other OSes. We believe this is probably due to the user have upgraded his system heavily during our
data sampling period, thus damaged the offset-based locality. In addition, the cloud will only have
more and more variations of operating systems with different installation configurations,
All these suggest data reduction base on offset may not work across different user snapshot backups.

\emph{Observation 3: There is not much duplication on user data if without multiple backups.}
For DDS data set, because it doesn't contain snapshot backups, the overall reduction ratio is only about 2:1,
which suggests we probably should put less effort on the reduction of user generated data. These data
are changed slowly and mostly by append, so locality should work well enough.


