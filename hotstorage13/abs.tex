\begin{abstract}
%Virtualization has became the engine behind many cloud computing platforms.
In a virtualized cloud cluster, frequent  snapshot backup of virtual disks improves
hosting  reliability; however, it  takes significant 
memory resource to perform data duplication in order to remove
excessive redundancy among snapshots.
This paper presents a low-cost deduplication solution scalable for a large number 
of virtual machines.  The key idea is to separate duplicate detection from 
the actual storage backup instead of using inline deduplication, and partition
global index and  detection requests among machines using  fingerprint values.
Then each machine conducts duplicate detection partition by partition independently with a minimal memory usage.
%in parallel among all machines and  each machine full duplication detection is 
Another optimization is to allocate and control
buffer space for exchanging  detection requests and duplicate summary among machines.
%The memory requirement and disk usage for the proposed solution is very small while the overall thoughput
%and backup process timing is not compromised. 
Our evaluation shows that  the proposed multi-stage scheme 
uses a small or modest amount of system resources while delivering a satisfactory backup throughput.
%in a large cloud setting.
%which only reqiires a very small amount of memory and CPU resource.  
%Experimental results  show the proposed scheme  can achieve high deduplication ratio while using
%a  small  amount of cloud resources. 

%Our system compares well with Amazon Glacier, in that, both of them are low-cost archival systems, 
%supporting lazy storage with asynchronous notification mechanisms and achieve parallelism by 
%reading/writing from multiple storage nodes/disks simultaneously. At the same time, 


%While dirtybit-based technique can identify unmodified data between versions, 
%full deduplication with fingerprint comparison  can remove more redundant content
%at the cost of computing resources.
%with  for similarity comparison and   reliability handling.
%Current snapshot deduplication is mainly done through copy-on-write 
%on fixed-size disk blocks. Such solutions cannot handle the
% cross VM data duplication because VMs do not share any data. 
%In addition, storing VM images and their snapshots
%in the same storage engine reduce the underline design flexibility because 
%these two kinds of data have distinct access requirements.
%In this paper, 
%we show that there is a large amount of duplicated data shared amongy virtual machines
%through a production VM data study and thus it is expective to perform cross-machine deduplication. 
% first perform a large scale study in production VM clusters 
%to show that cross VM data duplication is severe due to they have large amount of
%common data. Then our data analysis finds out that the overall data duplication pattern follows the Zipf's law.
%Base on these discoveries, we propose a snapshot storage deduplication scheme using variable-size chunking
%to address the above problem efficiently.
%We eliminate the majority of cross VM data duplication by pre-select
%a small set of frequently seen data blocks to be shared globally, and we also remove
%many cross snapshot duplication by using smaller chunking granuarity and locality.
\end{abstract}
%\begin{IEEEkeywords}
%Cloud storage backup,  Virtual machine snapshots,  Distributed data deduplication
%\end{IEEEkeywords}
