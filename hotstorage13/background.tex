
\section{Background and Related Work}
\label{sect:background}




In a virtualized cloud environment such as ones provided by Amazon EC2\cite{AmazonEC2} and Alibaba Aliyun\cite{Aliyun}, each instance of a guest operating system runs on a virtual machine, accessing virtual hard disks represented as virtual disk image files in the host operating system.
For VM snapshot backup, file-level semantics are normally not provided.
Snapshot operations are taken place at the virtual device driver level, which means no fine-grained file system metadata can be used to determine the changed data. Only raw access information at disk block level are provided. 
Each physical hosts many VMs and a cloud cluster has  petabytes of data that need to be backed up at daily basis, 
if not more frequently. But at the same time snapshot tasks must not affect the normal cloud service, 
which means that ideally only a very small slice of CPU and memory can be used for the backup purpose.


The previous work for storage backup has extensively used  data deduplication techniques can eliminate redundancy globally among different files from different users.
Backup systems have been developed to use content hash (finger prints) to identify duplicate
content~\cite{venti02,Rhea2008}.
%,NGmiddleware2011}.
Today's commercial data backup systems (e.g. from EMC and NetApp)
%\cite{emc_avamar}\cite{datadomain_whitepaper}
use a variable-size chunking algorithm to detect duplicates in file data~\cite{similar94,hydrastor09}.


Several techniques have been proposed to speedup searching of duplicate
content. For example,
Zhu et al.~\cite{bottleneck08} tackle it
by using an in-memory Bloom filter and prefetch groups of chunk IDs that are likely to be
accessed together with high probability. It takes significant memory resource for filtering and caching.
NG et al.~\cite{ NGmiddleware2011}  use
a related filtering technique for integrating deduplication in Linux  file system and the memory
consumed is up to 2GB for a single machine. That is still too big in our context discussed below.
Lillibridge et al.~\cite{sparseindex09} break list of chunks
into large segments, the chunk IDs in each incoming segment are sampled and the segment is
deduplicated by comparing with the chunk IDs of only a few carefully selected backed up segments.
These are segments that share many chunk IDs with the incoming segment with high probability.
Deepavali et al.~\cite{extreme_binning09}  use fingerprint-based file similarity  and group similar files into the same physical location (bins) to deduplicate against each other.
That leads  to a smaller amount of memory usage for storing meta data in fingerprint
lookup with a tradeoff of the reduced recall ratio.

Additional inline deduplications are studied in ~\cite{idedup}. All of the above appraoches have focused on
such inline duplicate detection, namely given a content chunk, how to detect if it is duplicate instantantly
to trigger the data backup of this chunk or not before the detection request for the next chunk can be launched. 
In our work, this constraint is relaxed and 
there is a waiting time for many duplicate detection requests. Our goal is to minimizing  computing  resource without
affeting the overall time of backup. 
