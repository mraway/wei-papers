\section{Conclusion}
\label{sect:final}

The contributing  of this work is a low-cost and scalable clustered deduplication solution
using a non-inline approach. Because of separation  of duplicate detection and actual backup,
we are able to evenly distribute  fingerprint comparison among clustered machine
nodes, and only load one partition at time at each machine for in-memory comparison.
That leads to an insignificant system resource, which  does not compete with the existing cloud services.
The tradeoff is that every machine has to read virtual machine images twice to accomplish backup. 
When the cluster size changes, our experiment also shows a linear speedup of overall throughputs
because highly parallel fingerprint comparisons.  The cluster  of 100 nodes can deliver about 
8.8GB per second deduplication performance.  
We currently assume each machine performs backup for all VMs hosted and in practice, the system only
needs to backup active VMs and thus the overall backup time would actually be much smaller.
Our future  work is to conduct more scalability studies and experiments with more production trace data.

%in parallel among all machines and  each machine full duplication detection is 
%Another optimization is to allocate and control
%buffer space for exchanging  detection messages and duplicate summary among machines.
%snapshot service in VM cloud. 
%Inner-VM deduplication localizes backup data dependency and exposes more parallelism  
%while cross-VM deduplication with a small common data set
%effectively  covers a large amount of duplicated data.
%Our solution accomplishes the majority of
%potential global deduplication saving while still meets stringent cloud resources requirement. 
%Evaluation using real user's VM data shows
%our solution can accomplish 75\% of what complete global
%deduplication can do. 
%Compare to today's widely-used snapshot technique, our scheme reduces almost
%two-third of snapshot storage cost.
%Finally, our scheme uses a very small amount of memory on each node, and leaves
%room for additional optimization we are further studying.


%The memory requirement and disk usage for the proposed solution is very small while the overall thoughput
%and backup process timing is not compromised. 
Our experiments show that  the proposed scheme 
uses a very small amount of system resources while accomplishing a satisfactory backup throughput
in a large cloud setting.
%which only reqiires a very small amount of memory and CPU resource.  
%Experimental results  show the proposed scheme  can achieve high deduplication ratio while using
%a  small  amount of cloud resources. 

%Our system compares well with Amazon Glacier, in that, both of them are low-cost archival systems, 
%supporting lazy storage with asynchronous notification mechanisms and achieve parallelism by 
%reading/writing from multiple storage nodes/disks simultaneously. At the same time, 


%While dirtybit-based technique can identify unmodified data between versions, 
%full deduplication with fingerprint comparison  can remove more redundant content
%at the cost of computing resources.
%with  for similarity comparison and   reliability handling.
%Current snapshot deduplication is mainly done through copy-on-write 
%on fixed-size disk blocks. Such solutions cannot handle the
% cross VM data duplication because VMs do not share any data. 
%In addition, storing VM images and their snapshots
%in the same storage engine reduce the underline design flexibility because 
%these two kinds of data have distinct access requirements.
%In this paper, 
%we show that there is a large amount of duplicated data shared amongy virtual machines
%through a production VM data study and thus it is expective to perform cross-machine deduplication. 
% first perform a large scale study in production VM clusters 
%to show that cross VM data duplication is severe due to they have large amount of
%common data. Then our data analysis finds out that the overall data duplication pattern follows the Zipf's law.
%Base on these discoveries, we propose a snapshot storage deduplication scheme using variable-size chunking
%to address the above problem efficiently.
%We eliminate the majority of cross VM data duplication by pre-select
%a small set of frequently seen data blocks to be shared globally, and we also remove
%many cross snapshot duplication by using smaller chunking granuarity and locality.




%\section*{Acknowledgments}
%{
%We would like to thank Weicai Chen and Shikun Tian from Alibaba for their kind support, 
%and the anonymous referees for their comments.
%Wei Zhang has received internship support from Alibaba  for VM backup system development.
%This work was supported in part by NSF IIS-1118106.
%Any opinions, findings, and conclusions or recommendations expressed in this material are those of the authors and
%do not necessarily reflect the views of Alibaba or the National Science Foundation.
%}
%Noted that 6\% is still significant, which is about 24GB per each VM and for a 1000 node Aliyun cluster,
%this is about 600 terabytes.
 
%Our experiments show th
%our solution can eliminate the majority of data duplication with a tiny fraction of
%block hash index store in memory. It does not only saves valuable system resouces in
%the VM cloud, but also makes deduplication much faster.
%
%
%Using  50 user VM data out of 1322 data disks as the training data and
%with  1.5\% as CDS threshold, we see the total 1198GB of new data is reduced by
%755.8GB, while perfect deduplication can reduce 1017.4GB. So 74.3\% of duplicate blocks are eliminated
%by pre-trained CDS, which is quite satisfactory.

