\begin{abstract}
A cloud environment that  hosts a large number of  virtual machines (VMs) has
a high storage demand for frequent backup of system image snapshots.
Deduplication of data blocks can lead a big reduction of redundant blocks when
their signatures are identical. However it is expensive and less fault-resilient to perform a global deduplication
using signatures and let a data block share by many virtual machines. 
This paper studies a VM-centric scheme which collocates a lightweight
backup service with other cloud services in a cluster and it integrates multiple duplicate detection strategies 
that  localize deduplication as much as possible within each virtual machine.
It also organizes the write of small data chunks into large file system blocks so
that each underlying file block is associated with one VM for most of cases.
Our analysis shows that  this VM centric scheme 
can  provide  better fault tolerance while using a small amount of computing and storage resource. 
This  paper  provides a comparative evaluation of this scheme in  accomplishing a high deduplication 
efficiency while sustaining a good backup throughput. 

%This paper studies the      a VM snapshot storage architecture which adopts multiple-level selective deduplication to bring the benefits of fine-grained data reduction into cloud backup storage systems.
%In this work, we describe our working snapshot system implementation, and provide
%early performance measurements for both deduplication impact and
%snapshot operations.
\end{abstract}
