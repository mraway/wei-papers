\begin{abstract}
A cloud environment that  hosts a large number of  virtual machines has
a high storage demand for frequent backup of system image snapshots.
Deduplication of data blocks can lead a big reduction of redundant blocks when
their signatures are identical. However it is expensive and less fault-resilient to perform a global comparison
of all data block signatures and let a data block share by many virtual machines. 
This paper studies a cluster-based design which collocates a lightweight
backup service with the cloud service and integrates multiple duplicate detection strategies that  
localize the deduplication as much as possible within each virtual machine.
Our analysis shows that  this virtual machine centric scheme uses a small amount of memory resource to conduct duplicate detection
and can  provide  better fault tolerance through deduplication localization.  
This  paper  provides a comparative evaluation of this scheme in  accomplishing a high deduplication 
efficiency while sustaining a good backup throughput. 

%This paper studies the      a VM snapshot storage architecture which adopts multiple-level selective deduplication to bring the benefits of fine-grained data reduction into cloud backup storage systems.
%In this work, we describe our working snapshot system implementation, and provide
%early performance measurements for both deduplication impact and
%snapshot operations.
\end{abstract}
