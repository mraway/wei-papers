\section{Conclusion}
\label{sect:final}
In this paper we propose a VM-centric deduplication scheme for 
VM snapshot backup in the cloud for maximizing fault isolation and tolerance. 
Inner-VM deduplication localizes backup data dependency and exposes more parallelism  
while cross-VM deduplication with a small common data set
effectively  covers a large percent of duplicate data.
The scheme organizes the write of small data chunks into large file system blocks so
that each underlying file block is associated with one VM for most cases.
by keeping most of the FSBs only referenced by 1 VM, we isolate faults and improve overal fault tolerance.
Our solution accomplishes reasonable and competitive deduplication efficiency while
still meeting stringent cloud resource usage requirements. 
Our analysis shows that  this VM centric scheme 
can  provide  better fault tolerance than VM-oblivious global deduplication schemes
while using a small amount of computing and storage resources. 


[Talk about more what we learn from Evaluation]
Evaluation using real user's VM data shows
our solution can accomplish 92\% of what complete global
deduplication can do. 
Compare to today's widely-used snapshot technique, our scheme reduces almost
two-third of snapshot storage cost.
Finally, our scheme uses a very small amount of memory on each node, and leaves
room for additional optimization we are further studying.


%This paper studies a cluster-based VM-centric scheme which collocates a lightweight
%backup service with the cloud service and it integrates multiple duplicate detection strategies that  
%localize the deduplication as much as possible within each virtual machine.
%This  paper  provides a comparative evaluation of this scheme in  accomplishing a high deduplication 
%efficiency while sustaining a good backup throughput. 

%This paper studies the      a VM snapshot storage architecture which adopts multiple-level selective deduplication to bring the benefits of fine-grained data reduction into cloud backup storage systems.
%In this work, we describe our working snapshot system implementation, and provide
%early performance measurements for both deduplication impact and
%snapshot operations.

\comments{
\section*{Acknowledgments}
{
We would like to thank Weicai Chen and Shikun Tian from Alibaba for their kind support, 
and the anonymous referees for their comments.
Wei Zhang has received internship support from Alibaba  for VM backup system development.
This work was supported in part by NSF IIS-1118106.
Any opinions, findings, and conclusions or recommendations expressed in this material are those of the authors and
do not necessarily reflect the views of Alibaba or the National Science Foundation.
}
}
%Noted that 6\% is still significant, which is about 24GB per each VM and for a 1000 node Aliyun cluster,
%this is about 600 terabytes.
 
%Our experiments show th
%our solution can eliminate the majority of data duplication with a tiny fraction of
%block hash index store in memory. It does not only saves valuable system resouces in
%the VM cloud, but also makes deduplication much faster.
%
%
%Using  50 user VM data out of 1322 data disks as the training data and
%with  1.5\% as CDS threshold, we see the total 1198GB of new data is reduced by
%755.8GB, while perfect deduplication can reduce 1017.4GB. So 74.3\% of duplicate blocks are eliminated
%by pre-trained CDS, which is quite satisfactory.

