\section{Architecture and Implementation Details}
\label{sect:architecture}
Our system runs on a cluster of Linux machines with Xen-based VMs.
A distributed file system (DFS) manages  the physical disk storage and we use 
an open source DFS called QFS~\cite{QFS}. 
All data needed for VM services, such as virtual disk images used by runtime VMs,
and snapshot data for backup purposes, reside in this distributed file system. 
One physical node hosts tens of VMs, each of which accesses its virtual machine disk image through the
virtual block device driver (called TapDisk\cite{Warfield2005} in Xen).

\subsection{ Components of a cluster node } 
As  depicted in Figure~\ref{fig:arch_vm}, 
there are four key service components running on each cluster
node  for supporting backup and deduplication: 
1) a virtual block device driver, 2) a snapshot deduplication component,
3) an append store client to store  and access snapshot data,
and 4)  a PDS client to support PDS index access. 

We use the virtual device driver in Xen that employs a bitmap to track the changes 
that have been made to virtual disk.
Every bit in the bitmap represents a fix-sized (2MB) region called a \textit{segment}, indicating whether the segment
has been modified since last backup. 
Segments are further divided into variable-sized chunks (average 4KB) 
using a content-based chunking algorithm~\cite{frame05}, 
which brings the opportunity of fine-grained deduplication.
When the VM issues a disk write, the dirty bit for the corresponding segment is set
and this indicates such a segments needs to be checked during snapshot backup. 
After the snapshot backup is finished, the driver resets the dirty bit map to a clean state.
[What happens with modification during snapshot backup stage.]

The representation of each snapshot has  a two-level index data structure.
The snapshot meta data (called snapshot recipe) contains a list of segments, each of which contains segment
metadata of its chunks (called segment recipe).
In snapshot and segment recipes, 
the data structures  includes reference pointers to the actual data location to eliminate the need for additional indirection.

\begin{figure*}[t]
    \centering
    \includegraphics[width=6in]{images/socc_arch_cluster}
    \caption{System Architecture}
    \label{fig:arch_vm}
\end{figure*}

\subsection{A VM-centric snapshot store for backup data}
\label{sect:store}
We build the snapshot storage on the top of a distributed file system.
Following the VM-centric idea for the purpose of fault isolation,
each VM has its own snapshot store, containing new data chunks which are considered
to be non-duplicates.
There is also a special store containing all PDS chunks shared among different VMs.
As shown in Fig.\ref{fig:as_arch}, we explain the data structure of snapshot stores as follows.

\begin{itemize}
\item Data of each VM snapshot store excluding PDS is divided into a set of containers and 
each container is approximately 1GB. 
The reason for dividing the snapshot into containers is to simplify the compaction process
conducted periodically. As discussed later, data chunks are deleted from old snapshots
and chunks without any reference from other snapshots can be removed by this compaction process.
By limiting the size of a container, we can effectively control the length of each round of compaction.
The compaction  routine can work on one container at a time and copy used data chunks to another container. 

Each container is further divided into a set of chunk data groups. Each chunk group is composed of
a set of data chunks and is the basic unit in data access and retrieval. 
In writing a chunk during backup, the system accumulates data chunks and store the entire
group as a unit after a compression.
When accessing a particular chunk, its chunk group is retrieved from the storage
and uncompressed. Given the high spatial locality and usefulness of prefetching  in 
snapshot chunk accessing~\cite{Guo2011,foundation08},
retrieval of  a data chunk  group naturally works well with prefetching. 
A  typical chunk group contains 100 to 1000 chunks, with an average size of 
200-600 chunks.

\item Each data container is represented by three files in the DFS:
1) the container data file holds the actual content, 
2) the container index file is responsible for translating a data reference
into its location within a container, and 
3) a chunk deletion log file records all the deletion requests within  the container.

\item A data chunk reference stored in the index of snapshot recipes
is composed of two parts: a container ID with 2 bytes and a local chunk ID with 6 bytes.
Each container maintains a local  chunk counter and assigns the current number 
as a chunk ID  when  a new chunk is added to this  container. 
Since data chunks are always appended to a snapshot store during backup, 
local chunk IDs are monotonically increasing.
When a snapshot chunk is to be accessed, the recipe for the snapshot will point a data chunk
in the PDS store or in a non-PDS VM snapshot  store. 
Using  a container ID, the corresponding container index file of this VM is accessed and 
the chunk group is identified using a simple chunk ID range search. Once the chunk group is loaded to memory, 
its header contains the exact offset of the corresponding chunk ID and the content is then accessed from the memory buffer.

\item The PDS chunks are a set of commonly used data and they are stored in one PDS file.
Since the total file size is relatively small, and
PDS data is re-calculated periodically, the PDS data file and its index are rebuilt completely. 
Each reference to a PDS data chunk in the PDS index is the offset within the PDS file, and the chunk size.

\end{itemize}

The snapshot  store supports three API calls.
\begin{itemize}
\item {\em Put(data)}. This places data chunk into the snapshot store and returns a reference to be stored in 
the recipe metadata of a snapshot. 
The write requests to append data chunks to a VM store are accumulated at the client side. 
When the number of write requests reaches a fixed group size, the snapshot store client compresses
the accumulated   chunk group, adds a chunk group index  to the beginning of the group, and then
appends the header and data  to the corresponding VM file.
A new container  index entry is also created for each chunk group and is written to the corresponding
container index file.
The writing of PDS data chunks is conducted periodically when there is a new PDS calculation.
\item{\em Get(reference)}.
The fetch operation for the PDS data chunk is straightforward since each reference contains 
the offset and size within the PDS  underlying  file.
We also maintain a small data cache for the PDS data service to speedup common data fetching.

To read a non-PDS chunk using its reference with container ID and local chunk ID,  the snapshot store client first loads the
corresponding VM's container index file specified by the container ID, then searches the chunk
groups  using their  chunk ID coverage.
After that, it reads the identified chunk group from DFS, decompresses it, and seeks to the exact chunk data 
specified by the chunk ID. 
Finally, the client updates its internal chunk data cache with the newly loaded content to 
anticipate future sequential reads.
\item {\em Delete(reference)}.
Chunk deletion occurs when a snapshot expires or gets deleted explicitly by a user
and we will discuss the snapshot deletion in detail in the following subsection.
When deletion requests are issued for a specific container,
those requests are simply recorded into the  container's deletion log initially and thus  a lazy
deletion strategy is exercised.
Once local chunk IDs appear in
the deletion log, they will not be referenced by any future snapshot and can be safely deleted when needed. 
Periodically, the snapshot  store picks those containers with an excessive
number of deletion requests to  compact and  reclaim the corresponding disk space. 
During compaction, the snapshot store creates a new container (with the same container ID) to replace the 
existing one. This is done by sequentially scanning the old container, copying all the chunks that are not 
found in the deletion log to the new container, and creating new chunk groups and indices. 
Every local chunk ID however is directly copied rather than re-generated. This
process leaves holes in the CID values, but preserves the order and IDs of chunks.
As a result, all data references stored 
in upper level recipes are permanent and stable, and the data reading process
is as efficient as before. Maintaining the stability of chunk IDs also ensures that recipes do not
depend directly on physical storage locations.
\end{itemize}

\begin{figure}[htbp]
  \centering
  \epsfig{file=images/sstore_arch, width=3in}
  \caption{Data structure of a VM snapshot store.}
  \label{fig:as_arch}
\end{figure}

\subsection{ VM-centric Approximate Snapshot Deletion with Leak Repair}
\label{sect:delete}

\begin{figure}[htbp]
  \centering
  \epsfig{file=images/deletion.png, width=3in}
  \caption{Approximate deletion}
  \label{fig:deletion_flow}
\end{figure}

In a busy VM cluster, snapshot deletions can occur frequently.
Deduplication complicates the deletion process because space saving relies on the sharing of data
and it requires the global reference of deleted chunks to be identified before  they can be safely removed.
While we can use the mark-and-sweep technique~\cite{Guo2011}, 
it still takes significant resources to conduct this process every time there is a snapshot deletion.
In the case of Alibaba, snapshot backup is conducted automatically and there are 
about 10 snapshot stored for every user. When there is
a new snapshot created every day,  there will be  a snapshot expired everyday to maintain
a balanced storage use. 

We seek a fast solution with low resource usage to delete snapshots.
Our VM-centric snapshot storage design simplifies the deletion process since 
we can focus on  unreferenced chunks within each VM.
The PDS data chunks are commonly shared among all VMs and we do not consider them
during snapshot deletion.  The selection of PDS data chunks is updated periodically independent of snapshot deletion process.
Another resource-saving strategy we propose is
an {\em approximate} deletion strategy to trade deletion accuracy for
speed and resource usage. Our method sacrifices a small percent of storage leakage
to efficiently identify unused chunks.

The algorithm contains three aspects.

\begin{itemize}
\item {\bf Computation for snapshot fingerprint summary.}
Every time there is a new snapshot created,
we compute a Bloom-filter with $z$ bits as the summary of reference pointers for all non-PDS chunks used 
in this snapshot. Given $h$ snapshots stored for a VM, there are $h$ summary vectors maintained.

\item {\bf Approximate deletion with fast summary comparison.}
When there is a snapshot deletion,  
we need to identify if  chunks to be deleted from that snapshot
are still used by other snapshots. 
This is done approximately and quickly by comparing the 
reference pointers of deleted snapshot with
the merged reference Bloom-filter summary of other live snapshots.
The merging of live snapshot Bloom-filter bits uses the logical OR operator 
and the merged vector still takes $z$ bits.
Since the number of live snapshots $h$ is limited for
each VM, 
the time and memory cost of this comparison is small, linear to the number of chunks to be deleted.

If a chunk's reference pointer is not found in the merged summary vector, we are sure that
this chunk is not used by any live snapshots, thus it's safe delete it. 
However, among all the chunks to be deleted, 
there is a small percentage of unused chunks  which
are misjudged as  being in use, resulting in a storage leakage.

\item {\bf Periodic repair of leakage}.
%[exlpain why second Bloom filter, why scan append store]
Leakage repair is conducted periodically to fix the above approximation error.
This procedure compares the live chunks for each VM with what are truly used through the VM snapshot metadata recipe.
That requires a scanning of all chunks in a VM; however it is a VM-specific procedure and thus
the cost is relatively small compared to the traditional mark-sweep procedure~\cite{Guo2011} which scans snapshot 
chunks from all VMs.
For example,
consider each reference pointer consumes 8 bytes plus  1 mark bit, a VM that has 40GB backup data with about
10 million chunks will need less than 90MB of memory to complete a VM-specific mark-sweep process.
\end{itemize}

%{\bf Discussion}
We now estimate the size of storage leakage and how often leak repair needs to be conducted,
given  a VM which keeps $h$ snapshots in the backup storage, and it creates and deletes one snapshot
everyday. Let $u$ be the total number of chunks brought by the initial backup, $\Delta u$ be the average
number of additional unique chunks added from one snapshot to the next snapshot version. Then the total number of unique
chunks used in a VM is about:
\[
U = u + (h-1)\Delta u.
\]

Each Bloom filter vector has  $z$ bits for each snapshot and let $j$ be the number of hash functions used by the
Bloom filter. The probablity that a particular bit is 0  in all $h$ summary vectors is  
$(1- \frac{1}{z}) ^{j U}$. Notice that a chunk may appear multiple times in these summary vectors; however, this should not 
increase the probability of being a 0 bit in all $h$ summary vectors.
Then the misjudgment rate of being in use $\epsilon$ is: 
\begin{equation}
\label{eq:falserate}
\epsilon = (1-(1-\frac{1}{z})^{jU})^j.
\end{equation}


For each snapshot deletion, the number of chunks to be deleted is nearly identical to the number of
newly add chunks $\Delta u$. 
Let $R$ be the total number of runs of approximate deletion between two consecutive 
repairs. we estimate  the total leakage $L$ after $R$ runs as:
\[
L = R \Delta u \epsilon
\]

When leakage ratio $L/U$ exceeds a pre-defined threshold $\tau$, we need to execute a leak repair:

\begin{equation}
\label{eq:leakrepair}
\frac{L}{U} = \frac{R \Delta u \epsilon}{u+(h-1)\Delta u } > \tau 
\Longrightarrow R > \frac{\tau}{\epsilon}\times\frac{u + (h-1)\Delta u}{\Delta u}
\end{equation}

For example in our tested dataset,  
each VM keeps $h=10$ snapshots and each snapshot has
about 1-5\% of new data. Thus $\frac{\Delta u}{u} \leq 0.05$. For a 40GB snapshot, $u\approx  10$ millions.
Then $U=10.45$ millions.
We choose  $\epsilon = 0.01$ and $\tau=0.1$.  From Equation~\ref{eq:falserate}, 
$z=10U=100.45$ million bits. From Equation~\ref{eq:leakrepair}, 
leak repair should be triggered once for every R=290 runs of approximate deletion. 
When one machine hosts 25 VMs and there is one snapshot deletion per day per VM, there would be 
only one full leak repair for one VM scheduled for every 12 days. Each repairs uses at most  90MB memory on average
as discussed earlier and takes a short period of time.

