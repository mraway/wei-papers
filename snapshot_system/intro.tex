\section{Introduction}
In a virtualized cloud environment such as ones provided by Amazon EC2\cite{AmazonEC2} and Alibaba Aliyun\cite{Aliyun},
each instance of a guest operating system runs on a virtual machine, accessing
virtual hard disks represented as virtual disk image files in the host operating system.
%virtual disk image files (e.g. .vhd, .vmdk) in the host operating system.
Because these image files are stored as regular files from the external point of view,
backing up VM's data is mainly done by taking snapshots of virtual disk images.

A snapshot preserves the data of a VM's file system at a specific point in time. 
VM snapshots can be  backed up  incrementally by comparing blocks from one version to another 
and only the blocks that have changed from the previous version of snapshot will be saved~\cite{Clements2009,Vrable2009}. 

Frequent  backup of VM snapshots increases  the reliability of VM's hosted in a cloud.
For example, Aliyun, the largest cloud service provider by Alibaba in China, 
provides automatic frequent backup of VM images to strengthen the reliability of its service for all users.
The cost of frequent backup of VM snapshots is  high because of the huge storage demand.
Using a backup service with full deduplication support~\cite{venti02,bottleneck08}
can identify content duplicates among snapshots to remove redundant storage content,  but the weakness is that it
either adds the  extra cost significantly or competes computing resource with the existing cloud services.
In addition, data dependence created by duplicate relationship among snapshots
adds the complexity in fault tolerance management, especially when  VMs can migrate around in the cloud. 

Unlike the previous work dealing with general file-level backup and deduplication, our problem is focused on 
virtual disk image backup. Although we treat each virtual disk  as a file logically, its size is very large.
On the other hand, we need to support parallel backup of a large number of virtual disks in a cloud every day. 
One key requirement we face at Alibaba Aliyun is that VM snapshot backup should only use a minimal amount of system
resources so that most of resources is kept for regular cloud system services or applications.
Thus our objective is to exploit the characteristics of VM snapshot data and
pursue a cost-effective deduplication solution. 
Another goal  is to decentralize VM snapshot backup and  localize  deduplication as much as possible,
which brings the benefits for increased parallelism  and fault isolation.

%,extreme_binning09,sparseindex09
By observations on the VM snapshot data from production cloud, we found snapshot data duplication 
can be easily classified into two categories: \emph{inner-VM} and \emph{cross-VM}. Inner-VM duplication
exists between VM's snapshots, because the majority of data are unchanged during each backup period. 
On the other hand, Cross-VM duplication is mainly due to widely-used software and libraries such as Linux and MySQL.
As the result, different VMs tend to backup large amount of highly similar data.

With these in mind, we  have developed a distributed multi-level solution to conduct 
segment-level  and block-level inner-VM  deduplication to localize the deduplication effort when possible.
It then makes cross-VM deduplication by excluding a small number of
popular common data blocks from being backed up. Our study shows that common data blocks
occupy significant amount of storage space while they only take
a small amount of resources to deduplicate.
Separating deduplication into multi levels effectively accomplish the major space saving goal
compare the global complete deduplication scheme, at the same time it makes
the backup of different VMs to be independent for better fault tolerance.

The following table shows the strength and weakness of some well-know deduplication systems:
\begin{table}
    \begin{tabular}{|l|l|l|l|}
        \hline
        ~           & Scalability & Low-cost & Full Dedup \\ \hline
        DDFS        & N           & N        & Y          \\ 
        Ex-bin      & Y           & Y        & N          \\ 
        Guo         & Y           & N        & N          \\ 
        iDedup      & N           & Y        & N          \\ 
        Founadation & N           & N        & Y          \\
        \hline
    \end{tabular}
\end{table}