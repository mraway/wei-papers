\section{Discussions and Conclusions}
\label{sect:conclusion}
%In this paper we propose a collocated backup service built on
%the top of a cloud cluster to reduce network traffic and infrastructure requirements.
The main contribution of this paper is the development and analysis of a low-profile and VM-centric deduplication scheme that
uses  a small amount of system resource in both snapshot deduplication and deletion
while delivering competitive deduplication efficiency.
The VM-centric design also allows an optimization for  better fault isolation and tolerance.
Our evaluation has the following results.
\begin{itemize}
\item {\em Competitive deduplication efficiency with a tradeoff}. 
The proposed VC scheme 
can accomplish over  96\% of what complete global
deduplication can do. In comparison, SRB~\cite{Dong2011,extreme_binning09}
can accomplish 97.86\% while 
the sampled index~\cite{Guo2011} can deliver 97\% for a different test dataset. 
Local similarity search removes 12.05\% of the original data after using a segment-based
 dirty bit method and among them, 3\% is contributed by the increased similarity search.
Notice that our earlier study~\cite{WeiZhangIEEE} reaches 92\% of what global deduplication can do
with a relatively simpler strategy and thus the new design in this paper makes a
VM-centric scheme much more competitive to a traditional VM-oblivious approach.
\item {\em Lower resource usage in deduplication}. 
The VC scheme achieves a 85K:1 ratio between raw snapshot data and memory usage
and is much more memory-efficient than 
SRB  with 4K:1 ratio and sampled index with 20K:1.
VC with 100 machines takes 1.31 hours to complete the backup of 2,500 VMs
using 19\% of one core and 16.7\% of IO bandwidth per machine. 
Processing time of VC is 23.9\% faster  than SRB in our tested cases
and the per-machine throughput of VC is reasonable based on the result 
from~\cite{Guo2011}. 
Noted that both VC and SRB are designed for a distributed cluster architecture while
the sampled index method is for a single-machine architecture.
\item {\em Higher availability}. 
%The availability of snapshots increases substantially when
%adding more replication for popular cross-VM file blocks and packaging chunks from the same 
%VM in one file system block. 
The snapshot availability of VC is 99.66\% with 3 failed machines in a 100-machine cluster
while it is 69.54\% for  SRB or a VM-oblivious approach.
The analysis shows that the replication degree
for the popular data set between 6 and 9 is good enough when the replication degree
for other data blocks is 3, and adds only a small cost to storage.
\end{itemize}
The offline PDS recomputation does require some modest I/O and memory resource and 
since the recomputing frequency is relatively low (e.g. on a monthly basis), we expect such resource
consumption is acceptable in practice.
%MSST-CHANGE-BEGIN
The erasure code has been shown to be effective  to improve reliability on duplicated data 
storage~\cite{Li2011} and such a technique can be incorported.
Other interesting future work includes
studying the physical layout of backup images~\cite{SmaldoneWH13} and 
further assessing PDS seperation.
% and comparing with related work~\cite{Vrable2009,Tan2010,Fu2014}. 
%MSST-CHANGE-END

%We have also developed a VM-centric strategy for snapshot deletion and the VM-centric design
%allows the deletion be localized for each VM, which greatly simplifies the reference counting
%process. That is beyond  the scope of this paper.

\comments{
\item {\em Lower cost in snapshot deletion}. 
The deletion in VC with summary vectors
takes less than a minute for snapshot deletion using 15MB memory per machine
and all machines can run this operation in parallel without data dependence. 
The grouped mark-sweep method can take 43 hours using
1.2 to 3GB memory per machine in our tested cases.
The leak repair in VC takes 0.82 hours using 50MB on average while processing
some VMs may use upto 1.96GB. Such a repair is conducted infrequently (e.g. every 19.6 days) 
}

%Currently we are studying how the 

%Similarity guided local search reduces cross-VM data dependency and exposes more parallelism  
%while global deduplication with a small common data set eliminates popular duplicates.
%VM-specific file block packing also enhances fault tolerance by reducing data dependencies.
%The design places a special consideration for low-resource usages as a collocated cloud service
%and 

