\begin{abstract}
Data deduplication has been widely used for cloud data backup
because of excessive redundant content blocks. Common techniques use dirty bits to detect
the version  change and  perform global fingerprint comparison to remove duplicates
across virtual machines. However, letting a duplicate data block be shared
by many virtual machines creates data dependence and is less fault-resilient. 
%This paper studies the impact of deduplication on 
%fault resilidence  of virtual machine snapshots, created by data dependence  among duplicates
%and proposes a VM-centric approach that strikes a balance between deduplication efficiency
%and fault isolation 
This paper proposes a VM-centric backup service on a converged  storage cluster architecture
and  strikes a balance for better fault isolation with competitive  deduplication efficiency.
It localizes deduplication as much as possible within each 
virtual machine, guided by similarity
search and associates  underlying file blocks with one VM for most cases.
It restricts global deduplication to popular chunks with extra replication support.
This VM-centric scheme also has  an advantage of   using less  computing resources for fingerprint comparison and simplifies snapshot deletion, suitable for a collocated backup service 
on a non-dedicated cluster architecture.
This  paper  describes an evaluation of this scheme to assess  its deduplication 
efficiency and fault resilience.

% Collocating a cluster-based duplicate service with other cloud services
%   necessary to eliminate
%redundant blocks and reduce cost. Collocating a cluster-based duplicate service with other cloud services

%A cloud environment that hosts a large number of virtual machines (VMs) has
%a high storage demand for frequent backup of system image snapshots. 
%Deduplication of data blocks is necessary to eliminate
%redundant blocks and reduce cost. Collocating a cluster-based duplicate service with other cloud services
%reduces network traffic;
%however, 
\end{abstract}
