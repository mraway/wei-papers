\section{Introduction}

Commercial ``Infrastructure as a Service'' clouds (i.e. {\em public
clouds}) often make use of commodity data center components to achieve the
best possible economies of scale.  In particular, large-scale e-commerce cloud
providers such as Google and Alibaba deploy ``converged'' components that
co-locate computing and storage in each hardware module (as opposed to having
separate computing and storage ``tiers.'')  The advantage of such an approach
is that all infrastructure components are used to support paying customers --
there are no resources specifically dedicated to cloud services.
%One of emerging architectures for building cloud services is a converged
%storage architecture that leverages commodity servers with software clustered
%storage.
%The converged architecture (server plus compute in same tier) grew out of the
%Google model. 
In particular, these providers use software to
aggregate multiple direct attached low-cost disks together across
servers as a way of avoiding the relatively high cost of network attached 
storage~\cite{googlefs03,hdfs10}.
%systems. %This(e.g. ~\cite{NutanixPaper}).  Leading cloud platform companies
%such as Google, Amazon, Alibaba, Microsoft Azure  have built a converged
%compute and storage infrastructure and used a distributed file system such as
%Google file system~\cite{googlefs03,hdfs10} to glue a large number of
%commodity servers.
% with local storage into a single cluster.
In such an environment,
%That  allows the use of scalable compute and storage, without incurring the
%costs and performance limitations associated with network storage.
each physical machine runs a number of virtual machines (VMs)
%as  instances of a guest operating system 
and their virtual disks are stored as disk image files, typically
using a local (possibly shared) file system.
Frequent snapshot backup of virtual disk images can
increase the service reliability by allowing VMs to restart from their latest
snapshot in the event of a server failure.  However, because the disk
blocks that make up the VMs are often the same from snapshot to snapshot
and are shared among VMs,  
deduplication of redundant content
blocks~\cite{venti02,bottleneck08} is necessary to substantially reduce the
storage demand.
%For example, the Aliyun cloud, which is  the largest cloud service provider
%by Alibaba in China, automatically conducts  the backup of virtual disk
%images to all active users every day.  The cost of supporting a large number
%of concurrent backup streams is high because of the huge storage demand and
%the use of deduplication

%Without deduplication,
A cloud may offload backup workload from production 
server hosts to dedicated backup proxy servers (e.g. EMC Data Domain) or
backup services (e.g. Amazon S3). This approach simplifies the cluster 
architecture and 
avoids potential performance degradation to production applications when 
backups are in progress.
However, sending out undeduplicated backup data wastes huge amount
of network bandwidth that would otherwise be available to user VMs. 
Even when the storage system implements deduplication internally
such dedicated backup storage is either difficult to scale out
(which can comprise the overall scalability of the converged architecture) 
or it must forego global deduplication in order to gain scalability 
(which increases backup cost).

This paper considers a backup service for converged cloud architectures
that is co-located with user VMs, sharing the cloud compute, network, and
storage resources with them.
%We describe methodologies to address the following two challenges: 
The key contribution of this work is a set of methodologies to address the following two challenges:
1) simple deduplication and snapshot deletion with competitive efficiency and 
a small resource usage without affecting other collocated cloud services;
2) improved fault isolation for the sharing of deduplicated data,
given that node failures in converged architecture are the norm rather than
exception. 
% so that users stop incurring storage charges immediately after snapshot deletion.
%\end{itemize}

% low-cost VM-centric snapshot backup approach that is co-located within 
% the existing converged storage architecture. Our solution leverages the scalability of the underlying
% distributed file system and provides global data deduplication in a resource friendly manner.

%For example, the Aliyun cloud, which is  the largest cloud service provider by Alibaba in China, 
%automatically conducts  the backup of virtual disk images to all active users every day.
%The cost of supporting a large number of concurrent backup streams is high
%because of the huge storage demand and the use of deduplication
 
%Using a separate  backup service with full deduplication support~\cite{venti02,bottleneck08}
%can effectively identify and remove content duplicates among snapshots, 
%but such a solution can be expensive. There is also a large amount of 
%network traffic to transfer  data from the host machines to the backup facility
%before duplicates are removed.

Version-based detection has been used to identify file blocks that have not
changed across snapshots so their redundant storage can be
avoided~\cite{Clements2009,Vrable2009,TanIPDPS2011}.
%the previous version of the
%snapshot~\cite{Clements2009,Vrable2009,TanIPDPS2011}, 
Alternatively, techniques that compare block ``fingerprints'' to
identify duplicates
that exist among all files regardless of their 
origin~\cite{Guo2011,Dong2011,extreme_binning09} have
also gained in popularity.  
One consequence of aggressive deduplication is the possible loss of failure
resilience.  Separate files share the same physical copies of blocks that are
logically duplicated among them.  If a shared block is lost, all files that
share that block are affected.  
%The problem is particularly acute for VM
%snapshots where the purpose of the snapshot is precisely to create redundant
%data copies so that VMs are more fault resilient.  
Thus, there exists a
tension between the additional redundancy that snapshot creation is intended
to provide, and the storage efficiency that deduplication makes possible.
The previous work on deduplication has not considered the impact of 
duplicate sharing on  fault tolerance.
%Because
%of highly repetitive content in snapshots from different VMs, many data chunks
%are shared by virtual machines.  Failure of a few shared data chunks can have
%a broad effect and snapshots of these virtual machines could be affected.  The
%previous work in deduplication focuses on the efficiency and approximation of
%fingerprint comparison, and has not addressed fault tolerance issues  together
%with deduplication.  

%This paper explores the storage and
%performance efficiencies that such a {\em VM-centric} 
%combination of local and global
%techniques can achieve.  
We term our approach {\em VM-centric} because the deduplication
algorithm considers VM boundaries in making its decisions as opposed to
treating all blocks as being equivalent within the storage system.
We detail the storage, performance,
and fault isolation properties
of our method and verify these properties empirically using a prototype
system that we have developed to run as a co-located back-up service in
converged IaaS cloud environments.
Our work focuses on sharing only ``popular'' data blocks 
(through deduplication and limited replication)
across 
virtual machines while using localized deduplication within each VM
to achieve both storage savings and fault isolation.
%, and to propose a VM-centric approach that localizes
%deduplication as much as possible and restricts global deduplication only to a
%limited set of most popular blocks.  
%In our method, local deduplication uses
%similarity-guided elimination to improve the deduplication coverage.  
Since the underlying storage file system block size is typically bigger than the average data chunk size
used for deduplication,  packaging
content from different VMs into the same file block  creates a ``false'' sharing that 
affects fault isolation. Thus we choose to avoid such VM-oblivious packaging.
%our approach uses a deduplication data chunk size that is smaller than a
%single file block and
%Thus our strategy is to packages data chunks from the same VM into a file
%system block as much as possible to improve fault isolation.  
%Deduplicating
%data locally within a VM also reduces the need to maintain a global index of
%chunks that are shared among VMs, thereby increasing the scaling efficiency of
%the snapshot system.  
The cost associated with localization, however, is in 
the lost
storage efficiency that cross-VM deduplication makes possible.  To recover
some of this efficiency we allow a small number of the most popular data
blocks to be deduplicated (and then replicated) across VMs.  


%Because data
%sharing is restricted, this VM-centric approach reduces the overall resource
%usage significantly during backup and simplifies the snapshot deletion
%process. This low-resource design is suitable when the backup service with
%deduplication is co-located with other services running on a shared compute
%and storage cluster.  We have evaluated this VM-centric approach using  a
%prototype system.
% that runs a cluster of Linux machines with Xen and a standard distributed
% file system for the backup storage. 
%This approach localizes duplicate detection within each VM  By narrowing
%duplicate sharing within a small percent of common data chunks and exploiting
%their popularity, this scheme can afford to allocate extra replicas of these
%shared chunks for better fault resilience while sustaining competitive
%deduplication efficiency.

%In addition, our VM-centric design allows garbage collection to be performed in a localized
%scope and we propose an approximate deletion scheme to reduce this cost further.
%Localization also brings the benefits of greater ability to exploit parallelism so
%backup operations can run simultaneously without a central  bottleneck.
%This VM-centric solution uses a small amount of  memory while delivering reasonable deduplication efficiency. 

%Another issue considered is that
%that garbage collection after deletion of old snapshots also competes for computing resources. 
%Sharing of data chunks among by multiple VMs needs to be detected during
%garbage collection and such dependencies complicate deletion operations. 

%************** Paper sections summary
%THIS NEEDS MODIFICATION
The rest of this paper is organized as follows.
Section~\ref{sect:background} reviews the background and discusses the  design options for snapshot backup 
with a VM-centric approach. 
Section~\ref{sect:deduplication}  analyzes the tradeoff and benefits of the VM-centric approach. 
Section~\ref{sect:architecture}  describes a system implementation that evaluates the proposed techniques.
%   the benefit of our approach for fault isolation. 
Section~\ref{sect:evaluation} is our experimental evaluation that compares with other approaches.
Section~\ref{sect:conclusion}  concludes this paper.
