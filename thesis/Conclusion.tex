\chapter{Conclusion and Future Works}
\label{chap:concl}
Deduplication in storage backup with an ever-increasing demand faces 
scalability and throughput challenges.  This dissertation investigates 
scalable techniques in building a VM snapshot storage system that provides low-cost
deduplication support for large VM clouds.  In particular, our work focuses on 
three specific  aspects  of VM snapshot deduplication: a synchronous solution for offline duplicate detection,
a multi-level selective approach for inline deduplication, and data management on the top of
cloud file system with an emphasis on fault tolerance and garbage collection.
Our system  has  been  implemented on Xen-based Linux VM clusters.

This dissertation begins with studying a low-cost multi-stage
parallel deduplication solution for synchronous backup. 
We split the original inline deduplication process into different stages to
facilitate parallel batch processing.
We first performing parallel duplicate detection for VM content blocks among all
machines before performing actual data backup. Each
machine accumulates detection requests and then performs 
detection partition by partition with a minimal resource usage. 
Fingerprint based partitioning allows
highly parallel duplicate detection and also simplifies
reference counting management.
Because the design separation
of duplicate detection and actual backup, we are able
to conduct distributed fingerprint comparison in parallel among clustered machine nodes, 
and only load one partition at time
at each machine for in-memory comparison.
This makes the proposed scheme very resource-friendly 
to the existing cloud services.

We have also presented a multi-level selective deduplication scheme for 
on-demand snapshot backup service. Our approach utilizes similarity guided inner-VM
deduplication to localize backup data dependence and exposes
more parallelism. Meanwhile cross-VM deduplication uses
a small popular data set to effectively cover a large amount of  content blocks. 
Our solution achieves the majority of what perfect
global deduplication can accomplish while meeting stringent 
resource requirements. Compared to the existing snapshot techniques, 
our scheme reduces about two-third
of snapshot storage cost. Finally, our scheme uses a very
small amount of memory on each node, and leaves room for employing other
optimization techniques. 

%The VM-centric backup service is collocated with other cloud servies.
%built on the top of a cloud cluster with emphasis on fault tolerance and garbage collection. 
%Benefited from our multi-level selective deduplication scheme, 
%we can reduce cross-VM data dependency and exposes more parallelism. 
The key contribution of our VM-centric data management is 
VM-specific file block packing that enhances fault tolerance by 
maximizing data localization and reducing cross VM sharing.
In addition, such a design allows garbage collection to be performed in a 
localized scope and we propose an approximate deletion
scheme to further reduce the cost.
As supported by theoretical analysis and experimental studies, 
the availability of VM snapshots increases substantially with a small
replication overhead for popular inter-VM chunks.
%Most importantly, our design places a special consideration for 
%low-resource usages as a collocated cloud service. 

The design of our VM-centric deduplication approach has a potential of being applied
to general purpose cloud backup. The idea of collecting
popular data becomes more interesting based on in data commonality among users.
However, the access to popular data could be a bottleneck in the I/O path if not designed carefully,
these issues remain to be solved.
