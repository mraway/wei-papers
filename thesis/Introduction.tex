\chapter{Introduction}
\label{chap:intro}
\section{Problem Statement and Motivation}
\label{intro:prob}
The ubiquity of the cloud computing has resulted in the 
widespread availability of cloud based services and applications accessible through 
the network. Examples include online storage services, big data analysis, and various websites.
In such a cluster-based cloud environment,
each physical machine runs a number of virtual machines as  instances of a guest operating system 
to contain different kind of user applications,
and their data are stored in virtual hard disks which are represented 
as virtual disk image files in the host operating system.
Frequent snapshot backup of virtual disk images is critical to increase  the service reliability
and protect data safty.
For example, the Aliyun cloud, which is  the largest cloud service provider by Alibaba in China, 
automatically conducts  the backup of virtual disk images to all active users every day.

The cost of supporting a large number of concurrent backup streams is high
because of the huge storage demand. 
If such VM snapshot data are plainly backed up without any duplicate reduction, storage waste would be 
extreme high. For example, a fresh Windows server 2008 installation costs 25 GB on the virtual disk,
and they are almost certianly duplicated with other Windows VM instances\cite{common07, pedal96}.
Using a separate  backup service with full deduplication support~\cite{venti02,bottleneck08}
can effectively identify and remove content duplicates among snapshots, 
but such a solution can be expensive. There is also a large amount of 
network traffic to transfer  data from the host machines to the backup facility
before duplicates are removed.

Unlike the previous work dealing with general file-level
backup and deduplication inside a centralized storage appliance, our problem is focused on virtual
disk image backup in a cluster of machines. Although we treat each virtual disk as a
file logically, its size is very large. On the other hand, we need
to support parallel backup of a large number of virtual disks
in a cloud every day. One key requirement we face at busy clouds
 is that VM snapshot backup should only use a minimal
amount of system resources so that most of resources is kept
for regular cloud system services or applications. Thus our
objective is to exploit the characteristics of VM snapshot data
and pursue a cost-effective deduplication solution. Another
goal is to decentralize VM snapshot backup and localize
deduplication as much as possible, which brings the benefits
for increased parallelism and fault isolation.

Despite its importance, storage deduplication remains a challenging task for cloud storage providers. 
And this is especially true for resource constrained large scale public VM cloud. 
In recognizing this importance and
the associated challenges,  my thesis is that

\textit{It  is  possible  to  build  an efficient, scalable, and fault tolerant backup storage
supporting highly accurate deduplication with sustainable high throughput
and easy management on deduplicated data.}

In  other  words,  the  main  goal  of this  study  is  to  answer  the  following 
question.  Given a large scale VM cluster running tens of thousand of VMs, 
how can such a snapshot backup service identify whether a piece of data in VM disk
really needs to backup? And how to manage backup data under complex sharing relationship?
This dissertation investigates 
techniques  in  building  a snapshot storage architecture that provides 
low-cost deduplication support for large VM clouds.  In particular,  it 
contains the following contributions to establish my thesis:

\begin{itemize}
\item A design of low-cost multi-stage parallel deduplication solution for automically batched deduplication.
\item An inline multi-level deduplication scheme with similarity guided local detection and popularity guided global detection.
\item The development and analysis of a VM-centric storage approach which considers fault isolation and integrates multiple duplicate detection strategies.
\item Fast and efficient garbage collection with approximate snapshot deletion and leak repair.
\end{itemize}

In general, this  dissertation  study  is  built  upon  a  large  body  of previous  research 
in storage deduplication and distributed storage systems.  The goal 
of this work is to provide low-cost deduplication support and storage management for 
large scale resource-constrained VM cloud. This solution should be collocated with existing
cloud infrastructure services and require no dedicated hardware support. It also needs to
accomplish good deduplication efficiency and does comprise the data reliability due to the
inherent volatility of commodity hardware.
Under the  above consideration, our system is designed to support multiple levels of deduplication
depending on backup requirements. The objective is to provide satisfactory deduplication efficiency
and high throughput with the emphasis on low resource usage and fault tolerancy.

\section{Dissertation Organization}
\label{intro:organ}
The rest of this dissertation is organized as follows.  
Chapter~\ref{chap:back} gives an introduction to the history and background of thesis problems.
Chapter~\ref{chap:overview} presents the overall VM snapshot storage architecture.
Chapter~\ref{chap:offline}  describes a synchronous and parallel deduplication scheme for offline batched complete deduplication.
Chapter~\ref{chap:inline} studies a multi-level selective deduplication solution for inline deduplication.
Chapter~\ref{chap:data}  describes a VM-centric storage management with approximate deletion.
Chapter~\ref{chap:concl}  concludes this  dissertation  and  briefly discusses some potential future work.

