\section{Introduction}

%1.how virtual machines work on storage
%2.cow and why they are not good
%3.introduce dedup
%1 

%Virtualization is the engine behind many popular cloud computing platforms.
%Amazon, Alibaba,  and many others have provided public VM clouds that host 
%tens of thousands of  virtual machines(VM) dynamically created
%for various active users. 
%In the  case of Alibaba cloud service, the sevice is called Aliyu, which
%provides the largest public cloud in based on the open-source Xen technology.
In a virtualized cloud environment such as ones provided by Amazon EC2 and Alibaba Aliyun,
each instance of a guest operating system runs on a virtual machine, accessing
virtual hard disks represented as virtual disk image files in the host operating system.
%virtual disk image files (e.g. .vhd, .vmdk) in the host operating system.
Because these image files are stored as regular files from the external point of view,
backing up VM's data is mainly done by taking snapshots of virtual disk images.

A snapshot preserves the data of a VM's file system at a specific point in time. 
VM snapshots can be  backed up  incrementally by comparing blocks from one version to another 
and only the blocks that have changed from the previous version of snapshot will be saved~\cite{Clements2009,Vrable2009}. 
%Even though the snapshots are saved incrementally, 
%when deleting a snapshot, only the data not needed for any other snapshot is removed. 
%So regardless of which prior snapshots have been deleted, all active snapshots will contain 
%all the information needed to restore the virtual disk.
%Snapshots can also be shared to instantiate multiple new virtual disks, such that user can make
%duplicate VMs with exact the same disk state. This feature is especially useful when a user wants to run
%an application on multiple VMs, for example Hadoop or MPI.

Frequent  backup of VM snapshots increases  the reliability of VM's hosted in a cloud.
For example, Aliyun, the largest cloud service provider by Alibaba in China, 
provides automatic frequent backup of VM images to strengthen the reliability of its service for all users.
The cost of frequent backup of VM snapshots is  high because of the huge storage demand.
Using a backup service with full deduplication support~\cite{venti02,bottleneck08}
can identify content duplicates among snapshots to remove redundant storage content,  but the weakness is that it
either adds the  extra cost significantly or competes computing resource with the existing cloud services.
In addition, data dependence created by duplicate relationship among snapshots
adds the complexity in fault tolerance management, especially when  VMs can migrate around in the cloud. 

Unlike the previous work dealing with general file-level backup and deduplication, our problem is focused on 
virtual disk image backup. Although we treat each virtual disk  as a file logically, its size is very large.
On the other hand, we need to support parallel backup of a large number of virtual disks in a cloud every day. 
One key requirement we face at Alibaba Aliyun is that VM snapshot backup should only use a minimal amount of system
resources so that most of resources is kept for regular cloud system services or applications.
Thus our objective is to exploit the characteristics of VM snapshot data and
pursue a cost-effective deduplication solution. 
Another goal  is to decentralize VM snapshot backup and  localize  deduplication as much as possible,
which brings the benefits for increased parallelism  and fault isolation.

%,extreme_binning09,sparseindex09
By observations on the VM snapshot data from production cloud, we found snapshot data duplication 
can be easily classified into two categories: \emph{inner-VM} and \emph{cross-VM}. Inner-VM duplication
exists between VM's snapshots, because the majority of data are unchanged during each backup period. 
On the other hand, Cross-VM duplication is mainly due to widely-used software and libraries such as Linux and MySQL.
As the result, different VMs tend to backup large amount of highly similar data.

With these in mind, we  have developed a decentralized multi-level solution to conduct 
segment-level  and block-level inner-VM  deduplication to localize the deduplication effort when possible.
It then makes cross-VM deduplication by excluding a small number of
popular common data blocks from being backed up. Our study shows that common data blocks
occupy significant amount of storage space while they only take
a small amount of resources to deduplicate.
Separating deduplication into multi levels effectively accomplish the major space saving goal
compare the global complete deduplication scheme, at the same time it makes
the backup of different VMs to be independent for better fault tolerance.

%Restricting the scope of global deduplication reduces
%the inter-component  data dependenc during machine failure.

%Several operations are provided for creating and managing snapshots and snapshot trees,
%such as create snapshots, revert to any snapshot, and remove snapshots.
%VM snapshots is different from traditional file system backup from a few aspects:
%\begin{enumerate}
%\item The backup loads are heavy but resources are limited. The whole VM cluster have PBs of data and need to be backuped at daily basis, if not more frequently. But at the same time snapshot tasks must not affect the normal VM operations, which means only a tiny slice of CPU and memory can be used for this purpose.
%\item No file system semantics are provided. Snapshot operations are taken place at the virtual device driver level, which means no fine-grained file system metadata can be used to determine the changed data. Only raw access information at disk block level are provided.
%\item Snapshots can be shared between users. This does not only complicates the parent-child relationship between snapshots, but also require all snapshots be managed in a global namespace.
%\end{enumerate}

%In this paper we intorduce the deduplication scheme of snapshot storage in Aliyun's VM cloud. We exploit the 
The rest of the paper is arranged as follows. 
Section ~\ref{sect:background}
discusses on some background and related work.
Section~\ref{sect:options} discusses the requirements and  design options.
Section ~\ref{sect:arch} presents our snapshot backup architecture with multi-level selective deduplication 
Section ~\ref{sect:exper} presents  our evaluation results on the effectiveness
of multi-level deduplication for snapshot backup. 
Section ~\ref{sect:final} concludes this paper.
% talks about Aliyun's VM snapshot system and deduplication process. In section 4 we present
%the snapshot deduplication results of our real user VMs.
