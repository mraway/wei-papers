\begin{abstract}
Virtualization has became the engine behind many cloud computing platforms.
In a virtualized computing environment, virtual machines
operate on virtual disks, so backing up user data is done by constantly 
taking snapshots of those virtual disks. Because such VM snapshots are huge in
terms of both quantity and individual sizes, snapshot storage would cost tremendous amount 
of space without deduplication.
Current snapshot deduplication is mainly done through copy-on-write 
on fixed-size disk blocks. Such solutions cannot handle the
 cross VM data duplication because VMs do not share any data. 
In addition, storing VM images and their snapshots
in the same storage engine reduce the underline design flexibility because 
these two kinds of data have distinct access requirements.

In this paper, we first perform a large scale study in production VM clusters 
to show that cross VM data duplication is severe due to they have large amount of
common data. Then our data analysis finds out that the overall data duplication pattern follows the Zipf's law.
Base on these discoveries, we propose a snapshot storage deduplication scheme using variable-size chunking
to address the above problem efficiently.
We eliminate the majority of cross VM data duplication by pre-select
a small set of frequently seen data blocks to be shared globally, and we also remove
many cross snapshot duplication by using smaller chunking granuarity and locality.
Experiment shows our design can achieve high deduplication ratio with very limited 
amount of resources in large scale VM cloud environment.
\end{abstract}